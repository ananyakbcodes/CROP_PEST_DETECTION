\documentclass[12pt]{article}

% Packages
\usepackage[a4paper,margin=1in]{geometry}
\usepackage{fancyhdr}
\usepackage{setspace}
\usepackage{titlesec}
\usepackage{hyperref}
\usepackage{graphicx}
\usepackage{xcolor}
\usepackage{minted} % for code formatting

% Page style
\pagestyle{fancy}
\fancyhf{}
\fancyhead[L]{Crop Disease Detection \& Prediction}
\fancyhead[R]{Cassiel L.}
\fancyfoot[C]{\thepage}

% Section styling
\titleformat{\section}{\large\bfseries\color{olive}}{}{0em}{}
\titleformat{\subsection}{\bfseries\color{teal}}{}{0em}{}

% Document
\begin{document}
\begin{titlepage}
\centering
\vspace*{2cm}
{\Huge\bfseries Individual Report \\[0.3em]
\textcolor{olive}{Crop Disease Detection and Prediction Project} \par}
\vspace{2cm}
{\Large\bfseries Member: Cassiel L. \\[0.5em]

Roll No.: \texttt{24P0051006}}
\vfill
{\large \textbf{Institute:} Goa University}
\end{titlepage}

\onehalfspacing

\section{Introduction}
The \textbf{Crop Disease Detection and Prediction} project aims to integrate weather-based data and image recognition to detect crop diseases early.  
This multi-phase system helps farmers monitor crop health and predict disease risk based on both \textit{environmental} and \textit{visual} cues.  
My contribution focused on the \textbf{Weather API Integration and Predictive Model Training} module.

\section{Objective}
The key objective of my component was to:
\begin{itemize}
    \item Collect real-time weather data using the \textbf{OpenWeatherMap API}.
    \item Process and label this data for machine learning.
    \item Train a \textbf{Random Forest Classifier} to predict disease-conducive conditions.
    \item Save and manage data within Google Drive and GitHub.
\end{itemize}

\section{Tools and Technologies Used}
\begin{itemize}
    \item \textbf{Language:} Python
    \item \textbf{Libraries:} pandas, requests, sklearn, joblib
    \item \textbf{Platform:} Google Colab linked with GitHub
    \item \textbf{API:} OpenWeatherMap (Real-time weather data)
\end{itemize}

\section{Methodology}

\subsection{Data Collection}
Weather data was fetched for six Indian cities: \textit{Panaji, Mumbai, Delhi, Chennai, Kolkata, and Pune}.  
The OpenWeatherMap API provided live data on temperature, humidity, rainfall, pressure, and wind speed.

A sample request looked like:
\begin{minted}[fontsize=\footnotesize, bgcolor=gray!5, frame=single]{python}
url = f"https://api.openweathermap.org/data/2.5/weather?q={city},I
N&appid={API_KEY}&units=metric"
r = requests.get(url).json()
\end{minted}

The conditions were then labeled:
\begin{quote}
If humidity $>$ 80\% and $20^\circ$C $\leq$ temperature $\leq 35^\circ$C and rainfall $>$ 2mm  
$\Rightarrow$ Label as \textbf{High Risk (1)}, else \textbf{Low Risk (0)}.
\end{quote}

\subsection{Model Training}
The collected dataset was stored as:
\begin{verbatim}
/content/drive/MyDrive/crop_project/data/weather_dataset.csv
\end{verbatim}

A Random Forest model was trained using:
\begin{minted}[fontsize=\footnotesize, bgcolor=gray!5, frame=single]{python}
X = df[["temp", "humidity", "rainfall", "pressure", "wind_speed"]]
y = df["label"]

model = RandomForestClassifier()
model.fit(X, y)
\end{minted}

The trained model was then saved as:
\begin{verbatim}
/content/drive/MyDrive/crop_project/models/weather_model.pkl
\end{verbatim}

\subsection{Integration}
The weather model forms the analytical foundation for environmental risk detection.  
Its predictions can be combined with the CNN-based image analysis module to form a hybrid prediction system.

\section{Results}
\begin{itemize}
    \item Successfully collected real-time weather data for multiple locations.
    \item Generated and saved dataset with seven key parameters.
    \item Trained and exported a reliable Random Forest model.
    \item Pushed model and dataset to GitHub under the \texttt{Cassiel} branch.
\end{itemize}

\section{Challenges Faced}
\begin{itemize}
    \item Handling API rate limits and missing data responses.
    \item Managing file paths between Colab, Drive, and GitHub.
    \item Ensuring consistent data formats during live collection.
\end{itemize}

\section{Conclusion}
The weather-based prediction model successfully identifies favorable conditions for crop diseases.  
When combined with the CNN visual detection model, the project offers a powerful and holistic system for smart agriculture.

\section{Individual Learning Outcomes}
\begin{itemize}
    \item Learned how to use and authenticate public APIs.
    \item Understood real-time data processing in machine learning workflows.
    \item Enhanced version control skills using Git and GitHub.
    \item Improved understanding of environmental influence on crop health.
\end{itemize}

\vfill
\begin{center}
\textbf{--- End of Report ---}
\end{center}

\end{document}

